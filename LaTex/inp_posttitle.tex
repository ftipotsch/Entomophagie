\chapter*{Eidesstattliche Erklärung}
Ich erkläre an Eides statt, dass ich die vorliegende Diplomarbeit selbst verfasst und keine anderen als die angeführten Behelfe verwendet habe. Alle Stellen, die wörtlich oder inhaltlich den angegebenen Quellen entnommen wurden, sind als solche kenntlich gemacht.
Ich bin damit einverstanden, dass meine Arbeit öffentlich zugänglich gemacht wird.

\vspace{1cm}
\begin{tabular}{c c c}
	& \hspace{4cm} & \\\cline{1-1}
	Ort, Datum & & \\
	\vspace{2cm}
	& & \\\cline{1-1}\cline{3-3}
	Leonid Hammer & & Kevin Glatz \\ 
	\vspace{2cm}
	& & \\\cline{1-1}
	 Florian Tipotsch
\end{tabular}

\chapter*{Abnahmeerklärung}
Hiermit bestätigt der Auftraggeber, dass das übergebene Produkt dieser Diplomarbeit den dokumentierten Vorgaben entspricht. Des Weiteren verzichtet der Auftraggeber auf unentgeltliche Wartung und Weiterentwicklung des Produktes durch die Projektmitglieder bzw. die Schule.

\vspace{1cm}
\begin{tabular}{c}
	\\\cline{1-1}
	Ort, Datum\\
	\vspace{2cm}
	\\\cline{1-1}
	Thorsten Schwerte
\end{tabular}	

\chapter*{Vorwort}
Wir wollen uns bei unseren Betreuern unserer Schule und bei unserem Projektpartner für die Unterstützung und die Zurverfügungstellung von Hilfsmittel, mit welcher wir diese Projekt abschließen konnten, bedanken.


\chapter*{Abstract (Deutsch)}
Heutzutage versuchen Menschen immer schonender mit der Welt umzugehen. Dabei werden auf viele Alternativen gegriffen, sei es bei Stromversorgung oder bei verarbeitung von Plastik. Wir haben unser Projekt jedoch auf die Nahrungsaufnahme konzentriert. Unser Projekt Entomophagie beim Menschen handelt von dem Züchten und dem Verzehr von Insekten. Dafür entwickeln wir einen Prototyp, von einem Brutkasten, welcher in Zukunft als
Basis für weitere Brutkästen dient. Das Ziel unseres Produktes ist eine Umweltfreundlichere
Nahrungsquelle zu erschließen die in der westlichen Welt fast komplett ungenutzt ist. Mithilfe des Brutkasten soll es jedoch für jeden möglich sein, Essen günstig und selber ’anzubauen’. Da Insekten aufgrund ihres niedrigen Energieverbrauchs, während der Züchtung, viel Umweltfreundlicher sind als, zum Beispiel Kühe. Zudem sind Insekten eine effizientere Eiweißquelle als Fleischprodukte. Dieser Prototyp ist mit einem Mikrocontroller ausgestattet, der es erlaubt Daten auszulesen, sowie selbstständig dafür sorgt das die Tiere gefüttert werden oder der Kasten belüftet wird. So läuft das Projekt darauf aus den Markt durch die Einfachheit und der weniger verschwenderischen Ernährung zu durchdringen, und Entomophagie beim Menschen auch in den westlichen Ländern voran zu bringen. Ebenso wird erklärt wie man die Tiere haltet und sie dazu bringt sich zu vermehren.


\chapter*{Abstract (Englisch)}
Our project Entomophagie is about growing and eating insects at home. To complete our task, we create a prototype of a breeding hub, which could be used for future adaptations of our model. The goal of our product is to show the western world the environmental friendly food source of insects, which is almost unused there. Furthermore we want to make it available for everyone to create their own food at home by simply breeding insects and eating them. The main reason insects are that more environmental friendly than, for example, cows is the energy consumption while breeding them.