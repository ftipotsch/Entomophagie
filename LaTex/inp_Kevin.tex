\def \currentAuthor {Kevin Glatz}
\newpage


\section{Verkablung}

Die Verkablung der einzelnen Module ist aus verschiedenen Gründen wichtig. Zum einen brauch jedes Modul genügend Strom um seine Aufgabe zu erfüllen, gleichzeitig würde es aber ohne die Erdung zu einem Kurzschluss kommen. Des weiteren ist es wichtig das die Module Daten auslesen und weitersenden können

\subsection{Aufbau}

(bild einfügen)

In Abbildung x können wir nun den Arduino und das Breadboard sehen.  Dort kann man direkt die drei Farben erkennen die jeweils für Stromzufuhr (blau), Datenauslesung (grün) und Erdung (braun) stehen.

(bild mit stromfluss einfügen)

Hier sieht man wie der Strom im Breadboard fließt. Das heißt wenn man in der Untersten Reihe das Board mit 5 Volt versorgt sind alle weiteren Steckplätze verwendbar. Wichtig hierbei ist, dass das Kabel für die Stromzufuhr auch auf der richtigen Reihe platziert wurde. Das erkennt man bei dieser Leiterplatte an dem Plus und Minus Zeichen Plus steht hierbei für Strom und Minus für die Erdung.

Falls man das Falsch verkabelt funktioniert das Modul nicht, es werden allerdings keine Schäden angerichtet


(Bild mit analog und digitalwerten einfügen)

Der Arduino hat zwei Möglichkeiten Daten auszulesen. Einmal spricht man hier von digitaler und analogen Datenauslesung. (internet suche) Digitale Werte bedeutet hier allerdings nur das der ausgelesene Wert als 0 oder 1 abgespeichert werden kann, analoge Daten sind bei diesem Punkt anders, da diese einen Wert zwischen 0 und x haben können (zitat)

\section{Programmierung}

Um die Daten der einzelnen Sensoren verwenden zu können, müssen diese ausgelesen und sinnvoll wiedergegeben werden. Nur sobald die Sensoren richtig Programmiert und verkabelt sind, kriegt man lesbare Information. Diese können anschließend weiter interpretiert und verwendet werden. 
In diesem Projekt kann der Quellcode in drei Bereiche eingeteilt werden. 



\subsubsection{Bibliotheken importieren und globale Variablen definieren}

In diesem Schritt importieren wir die nötigen Bibliotheken welche aus GitHub oder direkt von Arduino zur verfügung gestellt bekommen. Auch werden alle Variablen die sowohl in den Methoden Setup und Loop verwendet werden erstellt. 

\subsubsection{Setup}

Die Funktion Setup wird verwendet um einmalige Konfigurationen zu tätigen. Dazu zählen unter anderem Aufbau einer Verbindung, Kalibrierung sensibler Controller oder ähnlichen. Die hier getätigten Aktivitäten werden nur ausgeführt wenn der Arduino startet bzw. zurückgesetzt wird.

\subsubsection{Loop}

Diese Methode verarbeitet und gibt all unsere Werte aus. Besonders an der Klasse Loop ist, dass diese dauerhaft wiederholt wird. Es gibt keine maximale Durchlaufmenge und die Geschwindigkeit eines Durchlaufs wird von der Methode \"delay\" in Millisekunden definiert. Mit Hilfe dieser Eigenschaften, können wir zeitnahe alle Daten auslesen. Falls nun ein Wert eine Mindestgrenze kann sofort darauf reagiert werden. 

Eine genauere Beschreibung der einzelnen Module und deren Programmierung erfolgt in den kommenden Seiten.



\subsection{DHT11 \& CCS811}

Für den DHT 11 sowie dem CCS811 gibt es eine von Adafruit frei verwendbare Bibliothek. Mithilfe von dieser kann man ein Objekt erstellen und es mithilfe diesem Objektes die Momentane Temperatur/Luftfeuchtigkeit auslesen.



Der Parameter \#include fügt externe Klassen in unser Projekt ein und erlaubt es uns in das Objekt dht sowie ccs zu erstellen.
Des weitere können wir mit dem Parameter \#define relevante Daten festlegen. Ein solches Beispiel ist von welchem Pin Daten empfangen werden.

	\begin{lstlisting}
	
	delay(2000);
	if(ccs.available()){
		if(!ccs.readData()){
			Serial.print("CO2: ");
			Serial.println(ccs.geteCO2());
		}
		else{
			Serial.println("ERROR!");
			//while(1);
		}
	}
	\end{lstlisting}
	\cite{CCS811man}



Die Loop Funktion, wiederholt sich im vorgegebenen Rhythmus immer wieder und gibt keine Variablen zurück. Dort werden alle Daten ausgelesen und an das WLAN Modul weitergeschickt bzw. in diesem Fall wird es auf den Serial Monitor ausgegeben.

Die WENN abfrage prüft als erstes ob der CO2 Sensor kalibriert wurde oder nicht. Falls es gelungen sein sollte gibt es die Werte mit dem Befehl Serial.println(ccs.geteCO2()); im Serial Monitor aus.

\subsection{Hebel}

Das Joystick bzw. der Hebel konnte direkt abgelesen werden. Wichtig hierbei ist es das man nicht nur eine Achse im Setup verwendet sondern beide, ansonsten kommt es bei dem Loop zu einem Fehler und der Wert der benötigten Achse verändert sich nicht.


\subsection{SG90}

Die vom verwendeten Bibliotheken waren bei der Installierung der Arduino eigenen IDE direkt dabei. Die Servo kontrolliert die Luftzufuhr und muss daher Werte vom CCS811 wiederverwenden

\begin{lstlisting}

  if(ccs.geteCO2() <= co2Min){
	//move the micro servo from 0 degrees to 180 degrees
	for(;servoAngle < 180; servoAngle++) {       
		servo.write(servoAngle);              
		delay(10);
	} 
} if (ccs.geteCO2() > co2Min && servoAngle != 0){
		servo.write(45); 
		servoAngle = 0;
		Serial.println("RETURN");
}
\end{lstlisting}
\cite{SG90tut}

Die If Abfrage prüft ob CO2 einen Mindestwert (co2Min) unterschreitet. Falls das passiert  wird eine Schleife ausgeführt die den Servo 180° dreht. Diese 180° würde die Lüftungsklappe aufhalten. Falls der CO2 Wert wieder in einem akzeptablen Bereich liegt und die Position des Motors nicht null beträgt, wird die zweite Schleife aktiviert die den Motor zur Position 0 zurückbringt 


\subsection{photocell}


